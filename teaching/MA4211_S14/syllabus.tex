\documentclass[a4paper]{article}

\usepackage{url, fullpage, hyperref}
\usepackage{enumitem}

\title{Syllabus of Functional Analysis}
\date{}

\begin{document}
\maketitle

\begin{minipage}[t]{0.5\linewidth}
  \begin{itemize}[leftmargin=*]
  \item Lecture time: Tuesday, Friday 10:00--12:00
  \item Lecture room: Block S16, Room 03--17
  \item Tutorial time: Wednesday 12:00--13:00
  \item Tutorial room: Block S17, Room 04--04 
  \end{itemize}
\end{minipage}
\begin{minipage}[t]{0.4\linewidth}
  \begin{itemize}
  \item Lecturer and tutor: Dr Wang Dong
  \item Office: Block S17, Room 06--20
  \item Tel: 6516\ 2746
  \item Email: \href{mailto:matwd@nus.edu.sg}{\nolinkurl{matwd@nus.edu.sg}}
  \end{itemize}
\end{minipage}

\begin{itemize}
\item Prerequsite: MA3209
  
\item Course description: This course is for students who are majors in pure mathematics or who need functional analysis in their applied mathematics courses. The objective of the module is to study linear mappings defined on Banach spaces and Hilbert spaces, especially linear functionals (real-valued mappings) on $L^p$, $C[0,1]$ and some sequence spaces. In particular, the four big theorems in functional analysis, namely, Hahn-Banach theorem, uniform boundedness theorem, open mapping theorem and Banach-Steinhaus theorem will be covered. Major topics: Normed linear spaces and Banach spaces. Bounded linear operators and continuous linear functionals. Dual spaces. Reflexivity. Hanh-Banach Theorem. Open Mapping Theorem. Uniform Boundedness Principle. Banach-Steinhaus Theorem. The classical Banach spaces: $C^0$, $\ell^p$, $L^p$, $C(K)$. Compact operators. Inner product spaces and Hilbert spaces. Orthonormal bases. Orthogonal complements and direct sums. Riesz Representation Theorem. Adjoint operators.
  
\item Course centents:
  \begin{itemize}
  \item 
    Basics of normed vector spaces
  \item 
    Linear operators in normed vector spaces
  \item 
    Basics of Hilbert spaces
  \item 
    Linear operators in Hilbert spaces
  \end{itemize}
\item References:
  \begin{itemize}
  \item
    E.~Kreyszig, \emph{Introductory Functional Analysis with Applications}, Wiley, 1978.
    
    (We cover most of Chapters 2, 3, 4, and part of Chapters 7, 8, 9)
  \end{itemize}
\item Assessment:
  Assessment of students will be based on
  \begin{itemize}
  \item a one-hour test during lecture time (\emph{tentatively} on 14, March 2014), 25\%
  \item tutorial participation, 5\%,
  \item a two-hour final examination, 70\%.
  \end{itemize}
  \emph{Any student who is absent without a valid reason from an assessment will be given zero mark for that assessment.}
\end{itemize}
\end{document}
