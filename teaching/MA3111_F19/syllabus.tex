\documentclass[a4paper]{article}


\ifx\fiverm\undefined
  \newfont\fiverm{cmr5}
\fi 
\input pictex

\usepackage{amsmath}
\usepackage{amsfonts}
\usepackage{url, fullpage, hyperref}
\usepackage{enumitem}

\newcommand{\realR}{\mathbb{R}}

\title{Syllabus of Complex Analysis I}
\date{}

\begin{document}
\maketitle

\begin{minipage}[t]{0.5\linewidth}
  \begin{itemize}[leftmargin=*]
  \item Lecture time: Monday 14:00--16:00, Thursday 12:00--14:00
  \item Lecture venue: LT32 (Monday), LT34 (Thursday)
  \item Tutorial time: Monday 12:00--13:00, Wednesday 11:00--12:00, Friday 8:00--9:00
  \item Tutorial room: S17-0404 (Monday and Wednesday), S17-0405 (Thursday)
  \end{itemize}
\end{minipage}
\begin{minipage}[t]{0.4\linewidth}
  \begin{itemize}
  \item Lecturer and tutor: Dr Wang Dong
  \item Office: Block S17, Room 06--20
  \item Tel: 6516\ 2746
  \item Email: \href{mailto:matwd@nus.edu.sg}{\nolinkurl{matwd@nus.edu.sg}}
  \item Office hour: Wednesday 10:00-- \\ 11:00, Friday 9:00--11:00
  \end{itemize}
\end{minipage}

\begin{itemize}
\item Prerequsite: (MA1104 or MA2104 or MA1507) and (MA3110 or MA3110S)
  
\item Course description:
  
  This module is a first course on the analysis of one complex variable. In this module, students will learn the basic theory and techniques of complex analysis as well as some of its applications.
  
  Target students are mathematics undergraduate students in the Faculty of Science.

  Major topics: complex numbers, analytic functions, Cauchy-Riemann equations, harmonic functions, elementary functions, contour integrals, Cauchy-Goursat Theorem, Cauchy integral formulas, Taylor series, Laurent series, residues and poles, applications to computation of improper
integrals.
  
\item References:
  \begin{itemize}
  \item 
    J.~Brown and R.~Churchill, \emph{Complex Variables and it Applications}, 9th edition, McGraw Hill.
    
    (This is the main textbook, and we will cover most of Chapters 1-7.  List of supplementary exercises will be taken from the book, and students are strongly advised to buy it.

    We will cover Chapters 1--7, but not all sections therein. We cover everything on the lecture notes that is based on this book.)
  \item 
    L.~V.~Ahlfors, \emph{Complex Analysis}, 3rd edition, McGraw Hill.

    (This is a classic text by one of the leading experts. Beautifully written although it is generally considered difficult by undergraduates.)
  \item
    E.~B.~Saff and A.~D.~Snider, \emph{Fundamentals of complex analysis for Mathematics, Science and Engineering}, Prentice Hall.
    
    (Relatively easy to read, written mostly for an engineering audience.)
  \end{itemize}
\item Assessment:
  Assessment of students will be based on
  \begin{itemize}
  \item
    One final examination (November), two hours long - 60\% (closed book exam with one help sheet).
  \item 
    One mid-term test (date to be decided) - 20\% (closed book test with one help sheet).
  \item 
    Four sets of (marked) homework - 10\%.
  \item 
    Tutorial attendence - 5\%.
  \item
    In-class quizzes - 5\%.
  \end{itemize}
\item
  LumiNUS course website:

  LumiNUS contains course materials (Lecture notes, Tutorial question sheets, Homework sheets), announcements, discussion forum, etc.

  Course materials are downloadable in Files at the LumiNUS course website. You should visit LumiNUS regularly for updates.
\end{itemize}

\end{document}

\section*{Introduction}

In Calculus/Analysis, we studied:
\begin{itemize}
\item 
  Real numbers: $x \in \realR$.

\item 
  Real-valued functions of a real variable $x$:
  \begin{equation*}
    f(x): \realR \to \realR,
  \end{equation*}
  e.g., $f(x) = x^2 + 1$.

\item 
Differentiation:
\begin{equation*}
  \beginpicture
  % \setcoordinatesystem units <0.06in,0.01in>
  \setcoordinatesystem units <0.08in,0.014in>
  \setplotarea x from -5 to 30, y from -10 to 120
  \arrow <6pt> [.16,.6] from -10 0 to 28 0
  \arrow <6pt> [.16,.6] from 0 -10 to 0 120
  \put {\small $x$}[lb] <0mm,-0.5mm> at 29 0
  \put {\small $y$}[cb] <0mm,0.5mm> at 0 123
  \put {\small $x_o$}[ct] <0mm,-0.5mm> at 16 -3
  \setquadratic
  \plot 1 30 16 60 25 110 /
  \setlinear
  \plot 8  34 16 60  24  86 /
  \put {$\bullet$} at 16 60
  \setdashes<2pt>
  \plot 16 0 16 60 /
  \endpicture
\end{equation*}

\item 
  Integration: e.g., $\int_1^2 f(x) dx$.
  \begin{equation*}
    \beginpicture
    \setcoordinatesystem units <0.30in,0.22in>
    \setplotarea x from -3 to 12, y from -2 to 8
    \setquadratic \plot 1 2 3 2.2 5 3.3 /
    \plot 5 3.3 7 5.5 9 7 /
    \setlinear
    \arrow <6pt> [.16,.6] from -1 0 to 10.5 0
    \arrow <6pt> [.16,.6] from 0 -1 to 0 8
    \put { 0}[rt] <0mm,-1mm> at -0.2 0
    \put {\small $x$}[lc] <0mm,-1mm> at 10.7 0.2
    \put {\small $y$}[cb] <1mm,0mm> at 0 8.4
    \put {\small $1$}[lc] <0mm,-1mm> at 3 -0.3
    \put {\small $2$}[lc] <0mm,-1mm> at 7 -0.3
    \put {\small $y=f(x)$}[cb]
    <1mm,0mm> at 8 7.3
    \setdashes<1.5pt>\plot 3 0 3 2.2 /
    \plot 7 0 7 5.5 /
    \endpicture
  \end{equation*}
\end{itemize}


In Complex Analysis, we study

\begin{itemize}
\item 
  Complex numbers: $~z=x+iy \in\mathbb  C$, with $x,y\in\mathbb  R, \quad i=\sqrt{-1}$.

\item 
  Complex-valued functions of a complex variable $z$:
  \begin{equation*}
    f(z):\mathbb  C\rightarrow \mathbb C, 
  \end{equation*}
  e.g., $f(z)=z^2-1+i$.

\item 
  Differentiation:  We only consider ``analytic functions'' like polynomials, trignometric functions, etc, but not functions like $g(z) = z\bar{z}$.

\item 
  Integration: We only consider the integration of ``analytic functions'', and essentially we only consider the integral over a curve: $\int_C f(x) dx$, where $C$ is a curve from one point on the complex plane to another point.
  
  Thus in general, we will NOT (be able to) define definite integrals $\int_{z_1}^{z_2}f(z) dz$.
\end{itemize}

Complex analysis has widespread applications in:

\begin{itemize}
\item 
  Physics (electrostatics, heat flow, fluid mechanics);

\item 
  Engineering (aerodynamics, aircraft design);

\item 
  Other branches of mathematics such as:
  \begin{itemize}
  \item 
    Computing certain improper integrals, e.g.
    \begin{equation*}
      \int_{-\infty}^\infty\dfrac{\cos 3x}{x^2+2x+2} dx=   
    \end{equation*}
    
  \item 
    The Fundamental Theorem of Algebra: Every polynomial of degree $n\geq 1$ has at least one (complex) root.
  \end{itemize}
\end{itemize}
\end{document}



