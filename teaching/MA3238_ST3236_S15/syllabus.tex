\documentclass[a4paper]{article}

\usepackage{url, fullpage, hyperref}
\usepackage{enumitem}

\title{Syllabus for MA3238/ST3236: Stochastic Processes I}
\date{}

\begin{document}
\maketitle

\begin{minipage}[t]{0.5\linewidth}
  \begin{itemize}[leftmargin=*]
  \item Lecture time: Wednesday, Friday 12:00--14:00
  \item Lecture venue: LT21
  \item Tutorial time: Tuesday 12:00--13:00, 13:00--14:00 \\
    \phantom{Tutorial time:} Wednesday 9:00--10:00, 10:00--11:00 \\
    \phantom{Tutorial time:} Thursday 9:00--10:00, 10:00--11:00
  \item Tutorial room: Block S17, Room 04--05 
  \end{itemize}
\end{minipage}
\begin{minipage}[t]{0.4\linewidth}
  \begin{itemize}
  \item Lecturer and tutor: Dr Wang Dong
  \item Office: Block S17, Room 06--20
  \item Tel: 6516\ 2746
  \item Email: \href{mailto:matwd@nus.edu.sg}{\nolinkurl{matwd@nus.edu.sg}}
  \end{itemize}
\end{minipage}

\begin{itemize}
\item Prerequsite: \{MA1101 or MA1101R or MA1508 or GM1302\} and \{MA2216 or ST2131\}
  
\item
  This module introduces the concept of modelling dependence and focuses on discrete-time Markov chains. 
  
\item
  Major topics: discrete-time Markov chains, examples of discrete-time Markov chains, classification of states, irreducibility, periodicity, first passage times, recurrence and transience, convergence theorems and stationary distributions. 

\item References:
  \begin{itemize}
  \item
    G.~Grimmett and D.~Stirzaker, \emph{Probability and Random Processes}, Oxford, 3rd ed., 2001.
 
  \item
    R.~Durrett, \emph{Essentials of Stochastic Processes}, Springer, 2nd ed., 2012.

    (This book has a free version \url{http://www.math.duke.edu/~rtd/EOSP/eosp.html}).
  \end{itemize}
\item Assessment:
  Assessment of students will be based on
  \begin{itemize}
  \item An in-class test during lecture time (\emph{tentatively} on 6, March 2014), 35\%
  \item Homework and tutorial participation, 15\%
  \item Final examination, 50\%
  \end{itemize}
  \emph{Any student who is absent without a valid reason from an assessment will be given zero mark for that assessment.}
\end{itemize}
\end{document}
