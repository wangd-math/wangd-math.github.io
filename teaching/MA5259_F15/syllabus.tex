\documentclass[a4paper]{article}

\usepackage{url, fullpage, hyperref}
\usepackage{enumitem}

\title{Syllabus of Functional Analysis}
\date{}

\begin{document}
\maketitle

\begin{minipage}[t]{0.5\linewidth}
  \begin{itemize}[leftmargin=*]
  \item Lecturer: Dr Wang Dong
  \item Lecture time: Wednesday 9:00--12:00
  \item Lecture room: Block S17, Room 05--12
  \end{itemize}
\end{minipage}
\begin{minipage}[t]{0.4\linewidth}
  \begin{itemize}
  \item Office: Block S17, Room 06--20
  \item Tel: 6516\ 2746
  \item Email: \href{mailto:matwd@nus.edu.sg}{\nolinkurl{matwd@nus.edu.sg}}
  \end{itemize}
\end{minipage}

\begin{itemize}
\item Module website: \url{https://dl.dropboxusercontent.com/u/13947995/MA_5259/index.html}
  
\item Prerequisite: (MA2216 or ST2131) and (MA3207H or MA3207 or MA4262)
  
\item Course description: We will start with a review of basic concepts and results from measure theory, and then introduce the axiomatic formulation of probability spaces using measure theory. The focus of this course is on classic limit theorems for independent random variables. Topics to be covered include: probability space, random variables, characteristic functions, weak and strong laws of large numbers, central limit theorem, Lindeberg's theorem, Poisson limit theorem, infinitely divisible distributions, stable distributions, extreme value distributions, elementary large deviations. 
  
\item References:
  \begin{itemize}
  \item
    R.~Durrett, \emph{Probability: Theory and Examples}, Cambridge, 2010.
    
    (A free version is provided by the author on \url{https://www.math.duke.edu/~rtd/PTE/PTE4_1.pdf}. Although the latest available version is the fourth, and you can find it in our book store, I personally use the third version: \url{http://book.douban.com/subject/1797985/}.)
  \end{itemize}
\item Assessment:
  Assessment of students will be based on
  \begin{itemize}
  \item Final exam: 50\%.
  \item Homework: 50\%.
  \end{itemize}
  \emph{Any student who is absent without a valid reason from an assessment will be given zero mark for that assessment.}
\end{itemize}
\end{document}

%%% Local Variables:
%%% mode: latex
%%% TeX-master: t
%%% End:
