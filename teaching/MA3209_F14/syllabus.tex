\documentclass[a4paper]{article}

\usepackage{url, fullpage, hyperref}
\usepackage{enumitem}

\title{Syllabus of Mathematical Analysis III}
\date{}

\begin{document}
\maketitle

\begin{minipage}[t]{0.59\linewidth}
  \begin{itemize}[leftmargin=*]
  \item Lecture time: Monday, Thursday 16:00--18:00
  \item Lecture room: Block S16, Room 04--30
  \item Tutorial time: Tuesday 10:00--11:00 and Friday 14:00--15:00
  \item Tutorial room: Block S17, Roon 05--11 (T), 04--04 (F)
  \end{itemize}
\end{minipage}
\begin{minipage}[t]{0.4\linewidth}
  \begin{itemize}
  \item Lecturer and tutor: Dr Wang Dong
  \item Office: Block S17, Room 06--20
  \item Tel: 6516\ 2746
  \item Email: \href{mailto:matwd@nus.edu.sg}{\nolinkurl{matwd@nus.edu.sg}}
  \end{itemize}
\end{minipage}

\begin{itemize}
\item Prerequsite: MA3110 or MA3110S

\item Course description: this module is an introduction to analysis in the setting of metric spaces. There are at least two advantages by adopting this slightly abstract point of view. First of all, it helps to crystallize fundamental concepts and elucidate the roles they play in the theory. Secondly, it provides a unified framework for applications of the results and techniques of mathematical analysis. This module will cover the basic theory of metric spaces and sample applications to other areas of mathematics. Is is highly recommended to students majoring in pure mathematics and to those who are interested in applied mathematics with an analytical flavour.

\item Course centents: 
  \begin{itemize}
  \item Definition and examples of metric spaces
  \item Convergence of sequences
  \item Topology of a metric space
  \item Continuous mappings
  \item Completeness
  \item Principle of contraction mappings and applications
  \item Compactness
  \item Connectedness
  \item Sequences and series of functions
  \end{itemize}
\item References:
  \begin{itemize}
  \item A.~N.~Kolmogorov and S.~V.~Fomin, \emph{Elements of the Theory of Functions and Functional Analysis}, Dover, 1999. (Chapter 2)
  \item W.~Rudin, \emph{Principles of Mathematical Analysis}, $3^{\mathrm{rd}}$ edition, McGraw-Hill, 1976. (Chapters 2, 3, 4, 7)
  \item W.~R.~Parzynski and P.~W.~Zipse, \emph{Introduction to Mathematical Analysis}, McGraw-Hill, 1987. (Chapter 10)
  \end{itemize}
\item Assessment:
  Assessment of students will be based on
  \begin{itemize}
  \item a 1.5-hour test during lecture time (\emph{tentatively} on 6, October 2012), 25\%
  \item tutorial participation, 5\%,
  \item a two-hour final examination, 70\%.
  \end{itemize}
  \emph{Any student who is absent without a valid reason from an assessment will be given zero mark for that assessment.}
\end{itemize}
\end{document}
